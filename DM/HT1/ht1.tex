\documentclass{article}
\usepackage{cmap}
\usepackage[T2A]{fontenc}
\usepackage[utf8]{inputenc}
\usepackage[english, russian]{babel}
\usepackage[a4paper, left=10mm, right=10mm, top=12mm, bottom=15mm]{geometry}
\usepackage{mathtools,amssymb}
\usepackage{graphicx}

\newenvironment{task}{\begin{center}\fontsize{14}{14}\selectfont\bf}{\rm\fontsize{12}{12}\selectfont\end{center}}


\begin{document}
	\begin{center}
		Токмаков Александр, группа БПМИ165 \\
		Домашнее задание 1
	\end{center}
	
	\begin{task} 
		№1
	\end{task}
	\begin{center}
	\begin{tabular}{ccc}
		$ \begin{cases}
		y-x\leq 2 \\
		y \geq 0 \\
		x+2y \geq 4 \\
		x+3y \leq 10 \\
		y-4x \geq 4
		\end{cases} $
		& $\quad \Leftrightarrow \quad$ &
		$ \begin{cases}
		y \leq x+2 \\
		y \geq 0 \\
		y \geq - \frac{x}{2} + 2 \\
		y \leq - \frac{x}{3} + \frac{10}{3} \\
		y \geq 4x+4
		\end{cases} $
	\end{tabular}
	\end{center}
	Построим графики соответствующих функций (цифрами отмечены области, удовлетворяющие неравенству с таким номером):\\
	\begin{center} \includegraphics[width=15cm]{plot1c} \end{center}
	Как видно из графиков, область, удовлетворяющая первым четырём неравенствам, лежит ниже графика $y=4x+4$ т.е. не удовлетворяет пятому неравенству. Таким образом, система несовместна.\\

	%=======================================================================================================

	\begin{task} 
		№2
	\end{task}
	\begin{center}
	\begin{tabular}{ccc}
		$ \begin{cases}
		5x+3y \rightarrow max \\
		5x-2y \geq 0 \\
		0 \leq x \leq 5 \\
		x + y \leq 7 \\
		y \geq 0
		\end{cases} $
		& $\quad \Leftrightarrow \quad$ &
		$ \begin{cases}
		5x+3y \rightarrow max \\
		y \leq \frac{5x}{2} \\
		0 \leq x \leq 5 \\
		y \leq -x +7 \\
		y \geq 0
		\end{cases} $
	\end{tabular}
	\end{center}
	\begin{center} 
		$f(x, y) = 5x + 3y$, $\nabla f = (5, 3)$, т.е. $f$ возрастает по направлению вектора $(5, 3)$  \\
	\end{center}
	Линии уровня имеют вид:
	\begin{center} 
		$c = 5x + 3y$, \quad $y = \frac{5x}{3} + \frac{c}{3}$ \\
	\end{center}
	\newpage
	Построим графики соответствующих функций и выделим область, удовлетворяющую неравенствам: \\
	\begin{center} \includegraphics[width=18cm]{plot2c} \end{center}
	
	Найдём максимальное $c$, при котором график $y = \frac{5x}{3} + \frac{c}{3}$ будет иметь общую точку с множеством решений неравенств. Из графиков можно видеть, что этой точкой будет точка пересечения $x=5$ и $y=-x+7$:
	\begin{center}
		\begin{tabular}{ccccc}
			$ \begin{cases}
			x = 5 \\
			y = -x +7 \\
			\end{cases} $
			& $\quad \Rightarrow \quad$ &
			$ \begin{cases}
			x = 5\\
			y = -5+7=2
			\end{cases} $
			& $\quad \Rightarrow \quad$ &
			$c = 5x + 3y = 5\cdot 5 + 3\cdot 2 = 31$
		\end{tabular} \\
	\end{center}
	Таким образом, максимальное значение целевой функции равно $31$ и достигается в точке $(5,2)$\\

%=======================================================================================================

	\begin{task} 
		№3
	\end{task}
	\begin{center}
		\begin{tabular}{ccccccc}
			$ \begin{cases}
			x-y \rightarrow max \\
			2x+y+z=4 \\
			6x+y+2z \leq 4 \\
			x \geq 0 \\
			y \geq 0
			\end{cases} $
			& $\quad \Leftrightarrow \quad$ &
			$ \begin{cases}
			x-y \rightarrow max \\
			z=4-2x-y \\
			6x+y+2z \leq 4 \\
			x \geq 0 \\
			y \geq 0 
			\end{cases} $
			& $\quad \Leftrightarrow \quad$ &
			$ \begin{cases}
			x-y \rightarrow max \\
			6x+y+2(4-2x-y) \leq 4 \\
			x \geq 0 \\
			y \geq 0
			\end{cases} $
			& $\quad \Leftrightarrow \quad$ &
			$ \begin{cases}
			x-y \rightarrow max \\
			y \geq 2x+4 \\
			x \geq 0 \\
			y \geq 0
			\end{cases} $
		\end{tabular}
	\end{center}
	\begin{center} 
		$f(x, y) = x - y$, $\nabla f = (1, -1)$, т.е. $f$ возрастает по направлению вектора $(1, -1)$  \\
	\end{center}
	Линии уровня имеют вид:
	\begin{center} 
		$c = x - y$, \quad $y = x - c$ \\
	\end{center}
	\newpage
	Построим графики соответствующих функций и выделим область, удовлетворяющую неравенствам: \\
	\begin{center} \includegraphics[width=18cm]{plot3c} \end{center}
	
	Найдём максимальное $c$, при котором график $y = x-c$ будет иметь общую точку с множеством решений неравенств. Из графиков можно видеть, что этой точкой будет точка пересечения $x=0$ и $y=2x+4$:
	\begin{center}
		\begin{tabular}{ccccc}
			$ \begin{cases}
			x = 0 \\
			y = 2x +4 \\
			\end{cases} $
			& $\quad \Rightarrow \quad$ &
			$ \begin{cases}
			x = 0\\
			y = 0+4=4
			\end{cases} $
			& $\quad \Rightarrow \quad$ &
			$c = x - y = -4$
		\end{tabular} \\
	\end{center}
	Таким образом, максимальное значение целевой функции равно $-4$ и достигается в точке $(0,4)$\\
	
	%=======================================================================================================
	
	\begin{task} 
		№4
	\end{task}
	\begin{center}
		\begin{tabular}{ccccccc}
			$ \begin{cases}
			x-2y+3z=6 \\
			2x+z \leq 4 \\
			z+3y \geq 14 \\
			x \geq 0 \\
			y \geq 0
			\end{cases} $
			& $\quad \Leftrightarrow \quad$ &
			$ \begin{cases}
			x=2y-3z+6 \\
			2x+z \leq 4 \\
			z+3y \geq 14 \\
			x \geq 0 \\
			y \geq 0
			\end{cases} $
			& $\quad \Leftrightarrow \quad$ &
			$ \begin{cases}
			2(2y-3z+6)+z \leq 4 \\
			z+3y \geq 14 \\
			2y-3z+6 \geq 0 \\
			y \geq 0
			\end{cases} $
			& $\quad \Leftrightarrow \quad$  &
			$ \begin{cases}
			4y-5z+8 \leq 0 \\
			z+3y -14 \geq 0 \\
			2y-3z+6 \geq 0 \\
			y \geq 0
			\end{cases} $
		\end{tabular}
		Далее символом $\Leftrightarrow$ обозначается равносильность в смысле совместности/несовместности, а не множества решений.
	    \begin{tabular}{cccccccccc}
	    	$ \begin{cases}
	    	4y-5z+8 \leq 0 \\
	    	z+3y-14 \geq 0 \\
	    	2y-3z+6 \geq 0 \\
	    	y \geq 0
	    	\end{cases} $
	    	& $\hfill \Leftrightarrow \hfill$ &
	    	$ \begin{cases}
	    	y \leq \frac{5z}{4}-2 \\
	    	y \geq \frac{-z}{3} + \frac{14}{3} \\
	    	y \geq \frac{3z}{2} - 3 \\
	    	y \geq 0
	    	\end{cases} $
	    	& $\hfill \Leftrightarrow \hfill$ &
	    	$ \begin{cases}
	    	\frac{5z}{4}-2 \geq \frac{-z}{3} + \frac{14}{3} \\
	    	\frac{5z}{4}-2 \geq \frac{3z}{2} - 3 \\
	    	\frac{5z}{4}-2 \geq 0
	    	\end{cases} $
	    	& $\hfill \Leftrightarrow \hfill$ &
	    	$ \begin{cases}
	    	z \geq \frac{80}{19} \\
	    	z \leq 4 \\
	    	z \geq \frac{8}{5}
	    	\end{cases} $
	    	& $\hfill \Leftrightarrow \hfill$ &
	    	$ \begin{cases}
	    	4 \leq \frac{80}{19} \\
	    	4 \leq \frac{8}{5}
	    	\end{cases} $
	    \end{tabular}
	    \vspace{10px}
	\end{center}
	Полученные неравенства неверны, следовательно исходная система несовместна.\\
	
		
	%=======================================================================================================
	
	\begin{task} 
		№5
	\end{task}
	\begin{center}
		\begin{tabular}{cccccccc}
			$ \begin{cases}
			x+y \rightarrow max \\
			2x+y-z \geq -5 \\
			2x-4y+z \geq -3 \\
			7x+4y-z \leq 12 \\
			x-5y-z \leq -3
			\end{cases} $
			& $\hfill \Leftrightarrow \hfill$ &
			$ \begin{cases}
			t \rightarrow max \\
			x=t-y \\
			2x+y-z \geq -5 \\
			2x-4y+z \geq -3 \\
			7x+4y-z \leq 12 \\
			x-5y-z \leq -3
			\end{cases} $
			& $\hfill \Leftrightarrow \hfill$ &
			$ \begin{cases}
			t \rightarrow max \\
			2(t-y)+y-z \geq -5 \\
			2(t-y)-4y+z \geq -3 \\
			7(t-y)+4y-z \leq 12 \\
			(t-y)-5y-z \leq -3
			\end{cases} $
			& $\hfill \Leftrightarrow \hfill$  &
			$ \begin{cases}
			t \rightarrow max \\
			z \leq 2t-4+5 \\
			z \geq -2t+6y-3 \\
			z \geq 7t-3y-12 \\
			z \geq t-6y+3
			\end{cases} $
			& $\hfill \Leftrightarrow \hfill$
		\end{tabular}
		\begin{tabular}{ccccccc}
			$ \begin{cases}
			t \rightarrow max \\
			2t-4+5 \geq -2t+6y-3 \\
			2t-4+5 \geq 7t-3y-12 \\
			2t-4+5 \geq t-6y+3
			\end{cases} $
			& $\hfill \Leftrightarrow \hfill$ &
			$ \begin{cases}
			t \rightarrow max \\
			y \leq \frac{2t}{3} + \frac{2}{3} \\
			y \geq \frac{5t}{3} - \frac{13}{3} \\
			y \geq -\frac{t}{6} + \frac{1}{3}
			\end{cases} $
			& $\hfill \Leftrightarrow \hfill$  &
			$ \begin{cases}
			t \rightarrow max \\
			\frac{2t}{3} + \frac{2}{3} \geq \frac{5t}{3} - \frac{13}{3} \\
			\frac{2t}{3} + \frac{2}{3} \geq -\frac{t}{6} + \frac{1}{3}
			\end{cases} $
			& $\hfill \Leftrightarrow \hfill$  &
			$ \begin{cases}
			t \rightarrow max \\
			t \leq 5 \\
			t \geq -\frac{2}{5}
			\end{cases} $
		\end{tabular}
	\end{center}
	Таким образом, $t=5$. Найдём остальные неизвестные: \\
	\begin{center}
		$y = \frac{2\cdot 5}{3} + \frac{2}{3} = \frac{12}{3} = 4$ \\
		\vspace{10px}
		$x = t - y = 5 - 4 = 1$ \\
		\vspace{10px}
		\begin{tabular}{ccc}
			$ \begin{cases}
			2\cdot 1+4-z \geq -5 \\
			2\cdot 1-4\cdot 4+z \geq -3 \\
			7\cdot 1+4\cdot 4-z \leq 12 \\
			1-5\cdot 4-z \leq -3
			\end{cases} $
			& $\hfill \Leftrightarrow \hfill$ &
			$ \begin{cases}
			z \leq 11 \\
			z \geq 11 \\
			z \geq 11 \\
			z \geq -16\
			\end{cases} $
		\end{tabular}
	\end{center}
	Максимум целевой функции равен 5 и достигается в точке $(1, 4, 11)$.\\

	
	\begin{task} 
		№6
	\end{task}
	Пусть $a$, $b$, $c$ - объёмы материалов А, Б и В соответственно, тогда: \\
	\hspace{10px}	$1000a + 1200b + 12000c \rightarrow max$ \quad - максимальный доход \\
	\hspace{10px}	$0 \leq a \leq 40$, $0 \leq b \leq 30$, $0 \leq c \leq 20$ \quad - ограничения на имеющиеся объёмы \\
	\hspace{10px}	$a+b+c \leq 60$ \quad - ограничение на вместимость самолёта \\
	\hspace{10px}	$2a+b+3c \leq 100$ \quad - ограничение на грузоподъёмность самолёта 
	\begin{center}
		\begin{tabular}{ccccccc}
			$ \begin{cases}
			1000a + 1200b + 12000c \rightarrow max \\
			0 \leq a  \\
			a \leq 40 \\
			0 \leq b \\
			b \leq 30 \\
			0 \leq c \\
			c \leq 20 \\
			a+b+c \leq 60 \\
			2a+b+3c \leq 100
			\end{cases} $
			
			& $\hfill \Leftrightarrow \hfill$ &
			
			$ \begin{cases}
			t \rightarrow max \\
			a = \frac{t}{1000} - \frac{12b}{10} - 12c \\
			0 \leq a  \\
			a \leq 40 \\
			0 \leq b \\
			b \leq 30 \\
			0 \leq c \\
			c \leq 20 \\
			a+b+c \leq 60 \\
			2a+b+3c \leq 100
			\end{cases} $
			
			& $\hfill \Leftrightarrow \hfill$ &
			
			$ \begin{cases}
			t \rightarrow max \\
			0 \leq \frac{t}{1000} - \frac{12b}{10} - 12c  \\
			\frac{t}{1000} - \frac{12b}{10} - 12c \leq 40 \\
			0 \leq b \\
			b \leq 30 \\
			0 \leq c \\
			c \leq 20 \\
			\frac{t}{1000} - \frac{12b}{10} - 12c+b+c \leq 60 \\
			2(\frac{t}{1000} - \frac{12b}{10} - 12c)+b+3c \leq 100
			\end{cases} $
		\end{tabular} \\
		\begin{tabular}{cccccc}
			$ \begin{cases}
			t \rightarrow max \\
			b \leq \frac{t}{1200} -10c \\
			b \geq \frac{t}{1200} -10c - \frac{100}{3}  \\
			b \geq 0 \\
			b \leq 30 \\
			0 \leq c \\
			c \leq 20 \\
			b \geq -55c + \frac{t}{200} -300 \\
			b \geq -15c + \frac{t}{700} - \frac{500}{7} \\
			\end{cases} $
			
			& $\hfill \Leftrightarrow \hfill$ &
			
			$ \begin{cases}
			t \rightarrow max \\
			b \leq \frac{t}{1200} -10c \\
			b \leq 30 \\
			b \geq \frac{t}{1200} -10c - \frac{100}{3}  \\
			b \geq 0 \\
			b \geq -55c + \frac{t}{200} -300 \\
			b \geq -15c + \frac{t}{700} - \frac{500}{7} \\
			0 \leq c \\
			c \leq 20 \\
			\end{cases} $
			
			& $\hfill \Leftrightarrow \hfill$ &
			
			$ \begin{cases}
			t \rightarrow max \\
			\frac{t}{1200} -10c - \frac{100}{3} \leq \frac{t}{1200} -10c \\
			0 \leq \frac{t}{1200} -10c \\
			-55c + \frac{t}{200} -300 \leq \frac{t}{1200} -10c \\
			-15c + \frac{t}{700} - \frac{500}{7} \leq \frac{t}{1200} -10c \\
			
			30 \geq \frac{t}{1200} -10c - \frac{100}{3}  \\
			30 \geq 0 \\
			30 \geq -55c + \frac{t}{200} -300 \\
			30 \geq -15c + \frac{t}{700} - \frac{500}{7} \\
			
			0 \leq c \\
			c \leq 20 \\
			\end{cases} $
			
			& $\hfill \Leftrightarrow \hfill$
		\end{tabular} \\
		
		\begin{tabular}{ccccccc}
			$ \begin{cases}
			t \rightarrow max \\
			0 \leq \frac{100}{3} \\
			c \leq \frac{t}{12000} \\
			c \geq \frac{t}{10800} - \frac{20}{3} \\
			c \geq \frac{t}{8400} - \frac{100}{7} \\
			
			c \geq \frac{t}{12000} - \frac{19}{3}  \\
			c \geq \frac{t}{11000} - 6 \\
			c \geq \frac{t}{10500} - \frac{710}{105} \\
			
			c \geq 0 \\
			c \leq 20 \\
			\end{cases} $
			
			& $\hfill \Leftrightarrow \hfill$ &
			
			$ \begin{cases}
			t \rightarrow max \\
			\frac{t}{10800} - \frac{20}{3} \leq \frac{t}{12000} \\
			\frac{t}{8400} - \frac{100}{7} \leq \frac{t}{12000} \\
			\frac{t}{12000} - \frac{19}{3} \leq \frac{t}{12000} \\
			\frac{t}{11000} - 6 \leq \frac{t}{12000} \\
			\frac{t}{10500} - \frac{710}{105} \leq \frac{t}{12000} \\
			0 \leq \frac{t}{12000} \\
			
			20 \geq \frac{t}{10800} - \frac{20}{3} \\
			20 \geq \frac{t}{8400} - \frac{100}{7} \\
			20 \geq \frac{t}{12000} - \frac{19}{3}  \\
			20 \geq \frac{t}{11000} - 6 \\
			20 \geq \frac{t}{10500} - \frac{710}{105} \\
			20 \geq 0 \\
			\end{cases} $
			
			& $\hfill \Leftrightarrow \hfill$ &
			
			$ \begin{cases}
			t \rightarrow max \\
			t \leq 720000 \\
			t \leq 400000 \\
			\frac{19}{3} \geq 0 \\
			t \leq 792000 \\
			t \leq 568000 \\
			t \geq 0 \\
			
			t \leq 288000 \\
			t \leq 288000 \\
			t \leq 316000  \\
			t \leq 286000 \\
			t \leq 281000
			\end{cases} $
			
			& $\hfill \Leftrightarrow \hfill$ &
			
			$ \begin{cases}
			t \rightarrow max \\
			0 \leq t \leq 281000
			\end{cases} $
			
		\end{tabular} \\
	\end{center}
	Таким образом, $t = 281000$, найдём остальные переменные: \\
	\begin{center}
		\begin{tabular}{ccc}
		$ \begin{cases}
		c \leq \frac{281000}{12000}
		c \leq 20
		\end{cases} $
		& $\quad \Rightarrow \quad$ &
		$c = 20$
		\end{tabular} \\
		\begin{tabular}{ccc}
			$ \begin{cases}
			b \leq \frac{281000}{1200} -10\cdot 20  = \frac{205}{6}\\
			b \leq 30 \\
			\end{cases} $
			& $\quad \Rightarrow \quad$ &
			$b = 30$
		\end{tabular}\\
		$a = \frac{281000}{1000} - \frac{12\cdot 30}{10} - 12\cdot 20 = 5$
	\end{center}
	Максимум равен 281000 и достигается при $a = 5$, $b = 30$, $c = 20$.


\end{document}
