\documentclass{article}
\usepackage{cmap}
\usepackage[T2A]{fontenc}
\usepackage[utf8]{inputenc}
\usepackage[english, russian]{babel}
\usepackage[a4paper, left=10mm, right=10mm, top=12mm, bottom=15mm]{geometry}
\usepackage{mathtools,amssymb}
\usepackage{graphicx}

\linespread{1.4}

\newenvironment{task}{\begin{center}\fontsize{14}{14}\selectfont\bf}{\rm\fontsize{12}{12}\selectfont\end{center}}


\begin{document}
	\begin{center}
		Токмаков Александр, группа БПМИ165 \\
		Домашнее задание 5
	\end{center}
	
	\begin{task} 
		№10
	\end{task}
	Будем считать, что вероятность угадать напёрсток в некоторой игре - $p = \frac{1}{3}$, вероятность не угадать - $q = 1 - \frac{1}{3} = \frac{2}{3}$, все игры независимы. \\ Тогда вероятность выиграть $k-1$ раз подряд, а затем на $k$-тый раз проиграть $P(k) = \left(\frac{1}{3}\right)^{k-1}\cdot \frac{2}{3} = 2\left(\frac{1}{3}\right)^k$ \\
	Найдём вероятность того, что первый проигрыш случится на игре с номером $k \leq n$:
	\begin{center}
		$P(k \leq n) = \sum\limits_{k=1}^{n}P(k) = \sum\limits_{k=1}^{n}2\left(\frac{1}{3}\right)^k
		= \frac{\frac{2}{3}\left(1 - \left(\frac{1}{3}\right)^n\right)}{1 - \frac{1}{3}} 
		= 1 - \frac{1}{3^n}$
	\end{center}
	Таким образом, функция распределения $F(n) = 1 - \frac{1}{3^n}$, построим её график:\\
	\begin{center}
		\includegraphics[width=19cm]{plot1m}
	\end{center}
	



	
	
	\begin{task} 
		№11
	\end{task}
	Пусть вероятностное пространство - $\Omega = [0, 1]^2$ - множество точек квадрата со стороной 1, $P((x_0, y_0) \in A) = \iint\limits_{A}1dxdy$ - вероятность того, что равновероятно выбранная точка попала в некоторое подмножество $A$. \\
	Найдём вероятность того, что $min(x, y) \leq a$ при $a \in [0, 1]$:
	\begin{center}
		$A = \left\lbrace(x, y) \in \Omega \hspace{4px}|\hspace{4px} min(x, y) \leq a \right\rbrace
		= \left\lbrace(x, y) \in \mathbb{R}^2 \hspace{4px}|\hspace{4px} 0\leq x \leq a \hspace{4px}\bigwedge\hspace{4px} x \leq y \leq 1 \right\rbrace 
		\cup 
		\left\lbrace(x, y) \in \mathbb{R}^2 \hspace{4px}|\hspace{4px} y < x \leq 1 \hspace{4px}\bigwedge\hspace{4px} 0 \leq y \leq a \right\rbrace $ \\
		
		$P(min(x, y) \leq a) = P(A) = \iint\limits_A1dxdy 
		= \int\limits_0^a \left( \int\limits_x^1 1dy \right)dx  + \int\limits_0^a \left( \int\limits_y^1 1dx \right)dy 
		= \int\limits_0^a \left( 1 - x \right)dx  + \int\limits_0^a \left( 1-y \right)dy 
		= $\\$= 2 \int\limits_0^a \left( 1 - x \right)dx = 2(a - 0) - 2(\frac{a^2}{2} - 0) 
		= 2a - a^2$\\
	\end{center}
	
	Найдём вероятность того, что $|x-y| \leq a$ при $a \in [0, 1]$:
	\begin{center}
		$B = \left\lbrace(x, y) \in \Omega \hspace{4px}|\hspace{4px} |x-y| \leq a \right\rbrace
		=$\\$= \left\lbrace(x, y) \in \mathbb{R}^2 \hspace{4px}|\hspace{4px} 0\leq y-x \leq a \hspace{4px}\bigwedge\hspace{4px} 0\leq x \leq y \leq 1 \right\rbrace 
		\cup 
		\left\lbrace(x, y) \in \mathbb{R}^2 \hspace{4px}|\hspace{4px} 0 \leq y < x \leq 1 \hspace{4px}\bigwedge\hspace{4px} 0 \leq x-y \leq a \right\rbrace =$ \\ $
		= \left\lbrace(x, y) \in \mathbb{R}^2 \hspace{4px}|\hspace{4px} x \leq y \leq x+a \hspace{4px}\bigwedge\hspace{4px} 0 \leq x \leq y\leq 1 \right\rbrace 
		\cup 
		\left\lbrace(x, y) \in \mathbb{R}^2 \hspace{4px}|\hspace{4px} 0 \leq y < x \leq 1 \hspace{4px}\bigwedge\hspace{4px} y < x \leq y+a \right\rbrace =$ \\ 
		
		$= \left\lbrace(x, y) \in \mathbb{R}^2 \hspace{4px}|\hspace{4px} x \leq y \leq x+a \hspace{4px}\bigwedge\hspace{4px} 0 \leq x \leq 1 - a \right\rbrace 
		\cup \left\lbrace(x, y) \in \mathbb{R}^2 \hspace{4px}|\hspace{4px} x \leq y \leq 1 \hspace{4px}\bigwedge\hspace{4px} 1-a < x \leq 1 \right\rbrace 
		\cup $\\$ \cup 
		\left\lbrace(x, y) \in \mathbb{R}^2 \hspace{4px}|\hspace{4px} 0 \leq y \leq 1 - a \hspace{4px}\bigwedge\hspace{4px} y < x \leq y+a \right\rbrace 
		\cup 
		\left\lbrace(x, y) \in \mathbb{R}^2 \hspace{4px}|\hspace{4px} 1-a < y \leq 1 \hspace{4px}\bigwedge\hspace{4px} y < x \leq 1 \right\rbrace$
		\vspace{10px}
		
		$P(|x-y| \leq a) = P(B) = \iint\limits_A1dxdy 
		= \int\limits_0^{1-a} \left( \int\limits_x^{x+a} 1dy \right)dx + \int\limits_{1-a}^{1} \left( \int\limits_x^{1} 1dy \right)dx
		 + \int\limits_0^{1-a} \left( \int\limits_y^{y+a} 1dx \right)dy + \int\limits_{1-a}^{1} \left( \int\limits_y^{1} 1dx \right)dy = $ \\ \vspace{5px}
		 $= 2\int\limits_0^{1-a} \left( \int\limits_x^{x+a} 1dy \right)dx + 2\int\limits_{1-a}^{1} \left( \int\limits_x^{1} 1dy \right)dx
		 = 2\int\limits_0^{1-a} adx + 2\int\limits_{1-a}^{1} \left( 1-x \right)dx =$ \\
		 $=2a\left((1-a)-0\right) + 2\left(\left(1 - (1-a)\right) - \left(\frac{1}{2} - \frac{(1-a)^2}{2}\right)\right) 
		 = 2a\left(1-a\right) + \left(2a - 1 + (1-a)^2\right) = $ \\
		 $= 2a(1-a) + (2a - 1 + 1 - 2a + a^a) = 2a(1-a) + a^2 = 2a - a^2$
	\end{center}
	Внезапно, $P(min(x, y) \leq a)= 2a - a^2 = P(|x-y| \leq a)$, т.е. распределения случайных величин $min(x, y)$ и $|x-y|$ совпадают.
	
\end{document}
