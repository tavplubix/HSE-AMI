\documentclass{article}
\usepackage{cmap}
\usepackage[T2A]{fontenc}
\usepackage[utf8]{inputenc}
\usepackage[english, russian]{babel}
\usepackage[a4paper, left=10mm, right=10mm, top=12mm, bottom=15mm]{geometry}
\usepackage{mathtools,amssymb}
\usepackage{graphicx}

\linespread{1.4}

\newcommand{\impl}{\quad\Leftrightarrow\quad}
\newcommand{\rimpl}{\quad\Rightarrow\quad}

\newenvironment{task}{\begin{center}\fontsize{14}{14}\selectfont\bf}{\rm\fontsize{12}{12}\selectfont\end{center}}


\begin{document}
	\begin{center}
		Токмаков Александр, группа БПМИ165 \\
		Домашнее задание 6
	\end{center}
	
	\begin{task} 
		№12
	\end{task}
	Найдём площадь $S$, которую робот успеет исследовать за время $t$:
	\begin{center}
		$S(t) = \frac{\pi R^2}{2} + 2R\cdot vt + \frac{\pi R^2}{2} = \pi R^2 + 2Rvt$
	\end{center}
	Будем считать, что неразорвавшиеся снаряды распределены по местности равномерно. Всего на просканированной местности окажется $N = \left[\lambda S\right] = \left[\lambda R(\pi R + 2vt)\right]$ снарядов. Каждый из них может быть обнаружен с вероятностью $p(v)$ и не обнаружен с вероятностью $1 - p(v)$, по схеме Бернулли вероятность $k$ успехов равна $P(k, t) = C_N^k\cdot p^k(v) \cdot (1 - p(v))^{N-k} = $\\ 
	$= C_{\left[\lambda R( \pi R + 2vt)\right]}^k\cdot p^k(v) \cdot (1 - p(v))^{\left[\lambda R(\pi R + 2vt)\right]-k}$ \\
	Вероятность обнаружить хотя бы один снаряд равна $P(\geq 1, t) = 1 - P(0, t)$, где $P(0, t) = (1 - p(v))^{\left[\lambda R(\pi R + 2vt)\right]}$ это вероятность не обнаружить ни одного снаряда. При $p(v) = e^{-\alpha v}$:
	\begin{center}
		$P(\geq 1, t) = 1 - (1 - e^{-\alpha v})^{\left[\lambda R(\pi R + 2vt)\right]} \approx 1 - (1 - e^{-\alpha v})^{\lambda R(\pi R + 2vt)}$ \\
		$\frac{\partial P(\geq 1, t)}{\partial v} = -\ln\left(1 - e^{-\alpha v} \right)\lambda R(\pi R + 2vt)  \cdot (1 - e^{-\alpha v})^{\lambda R(\pi R + 2vt)} 
		\cdot 
		\left( \frac{\alpha e^{-\alpha v}}{1 - e^{-\alpha v}} \cdot \lambda R(\pi R + 2vt) + \ln\left(1 - e^{-\alpha v} \right) \cdot \lambda R2t \right) = 0$ \\
		$\impl 
		\left[\begin{array}{l}
			1 - e^{-\alpha v} = 0 \\
			\frac{\alpha e^{-\alpha v}}{1 - e^{-\alpha v}} \cdot \lambda R(\pi R + 2vt) + \ln\left(1 - e^{-\alpha v} \right) \cdot \lambda R2t = 0
		\end{array} \right. 
		\impl
		\left[\begin{array}{l}
		v = 0 \\
		\frac{\alpha e^{-\alpha v}}{1 - e^{-\alpha v}} \cdot (\pi R + 2vt) = \ln\left(e^{-\alpha v} -1\right) \cdot 2t
		\end{array} \right. $
	\end{center}
	
\end{document}
