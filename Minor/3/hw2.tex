\documentclass{article}
\usepackage{cmap}
\usepackage[T2A]{fontenc}
\usepackage[utf8]{inputenc}
\usepackage[english, russian]{babel}
\usepackage[a4paper, left=10mm, right=10mm, top=12mm, bottom=15mm]{geometry}
\usepackage{mathtools,amssymb}
\usepackage{graphicx}
\usepackage{setspace}

\usepackage{listings}

\newenvironment{task}{\begin{center}\fontsize{14}{14}\selectfont\bf}{\rm\fontsize{12}{12}\selectfont\end{center}}

\newcommand{\tch}{\hspace{4px}|\hspace{4px}}
\newcommand{\impl}{\ \Leftrightarrow \ }
\newcommand{\rimpl}{\ \Rightarrow \ }
\newcommand{\res}[3]{\begin{array}{lcr} #1 & & #2 \\ \hline & #3 \end{array}}
\newcommand{\com}{, \hspace{5px}}
\newcommand{\N}{\mathbb{N}}
%\newcommand{\rimpl}{\ \Rightarrow \ }

\begin{document}
	\begin{center}
		Токмаков Александр, группа БПМИ165 \\
		Домашняя работа 2
	\end{center}

%==============================================================================================
	
	\begin{task} 
		№1
	\end{task}

	Нельзя утверждать, что $SAT \not\in P$ только потому, что приведён экспоненциальный алгоритм решения. Возможно, существует полиномиальный алгоритм (нужно доказать, что не существует такого алгоритма).
	
%==============================================================================================
	
		
	\begin{task} 
		№2
	\end{task}

%	Похоже, в задании опечатка: должно быть $coNP = \lbrace L \ | \ \overline{L} \in NP \rbrace$ вместо $coNP = \lbrace L \ | \ L \in NP \rbrace$. \\
	
	$SAT \leq_p \overline{TAUT}$ (если отрицание формулы выполнимо, то она не тавтология и наоборот) и $SAT \in NPC \ \Rightarrow\ $ \\ $\overline{TAUT} \in NPC \subset NP \ \Rightarrow\ TAUT \in coNP$. Более того, $TAUT \in coNPC$ т.к. если некоторый язык $L \in coNP$, то $\overline{L} \in NP$ и тогда $\overline{L} \leq_p \overline{TAUT} \ \Rightarrow \ L \leq_p TAUT$ т.е. любой $coNP$ язык сводится к $TAUT$. 
	
	Пусть $3SAT \leq_p TAUT$ и $TAUT \leq_p 3SAT$. Возьмём некоторый язык $L \in NP$: \\ $3SAT \in NPC \rimpl L \leq_p 3SAT \leq_p TAUT \rimpl L \in coNP \rimpl NP \subset coNP$.\\ Аналогично для некоторого языка $L \in coNP$: \\ $TAUT \in coNPC \rimpl L \leq_p TAUT \leq_p 3SAT \rimpl L \in NP \rimpl coNP \subset NP$. \\
	Таким образом, $NP = coNP$.
	
	Пусть $NP = coNP$. Тогда $TAUT \in coNP = NP \rimpl TAUT \leq_p 3SAT$ т.к. $3SAT \in NPC$. Аналогично $3SAT \in NP = coNP \rimpl 3SAT \leq_p TAUT$ т.к. $TAUT \in coNPC$. 

%==============================================================================================
	
	
	\begin{task} 
		№3
	\end{task}
	
	Не верно, т.к. в таком случае на словах из $\overline{L}$ машина Тьюринга может работать более чем полиномиальное время или вообще не остановиться т.е. $\overline{L}$ не разрешим за полиномиальное время, значит $L$ тоже не разрешим за полиномиальное время.
	
%==============================================================================================
		
	\begin{task} 
		№4
	\end{task}
	
	Рассмотрим алгоритм разрешения $T$:
	\begin{lstlisting}
	def inT(n):
		m = 1
		while m < n:
			m = m * 3
		return m == n
	\end{lstlisting}
	Все операции (умножение на 3 и сравнения) выполняются за полиномиальное время от $len(m)$ (длины слова $m$), $len(m) \leq len(3n) \leq len(n) + 1$. Цикл выполнится ровно $k = \lceil \log_3(n) \rceil = \lceil \frac{\log_{10}(n)}{\log_{10}(3)} \rceil$ раз, $\log_{10}(n) \leq len(n)$. Таким образом, алгоритм работает за полиномиальное время от длины входа.

	
	
%==============================================================================================
	
	
			
	\begin{task} 
		№5
	\end{task}

	Если в графе есть гамильтонов цикл, то последовательность вершин, образующих этот цикл, будет сертификатом: длина такого сертификата полиномиальна от длины входного слова (длина цикла равна числу вершин) и можно за полиномиальное время проверить, что такой цикл действительно есть в графе просто пройдя по нему. И наоборот: если существует сертификат -- такая последовательность не повторяющихся вершин, что каждая из них смежна со следующей и последняя смежна с первой, то в графе очевидно есть гамильтонов цикл.
	



%==============================================================================================	


	\begin{task} 
		№6
	\end{task}

	Если таблица как-то заполнена, то можно за полиномиальное время проверить, что она заполнена нужным образом: для каждой из $n^4$ клеток проверить, что такого числа нет в строке ($n^2$ клеток), столбце ($n^2$ клеток) и малом квадрате ($n^2$ клеток). Правильное заполнение таблицы будет сертификатом: оно полиномиально от размера входа и оно же будет решением судоку. Таким образом, это задача класса $NP$, значит $SUDOKU \leq_p SAT$ т.к. $SAT \in NPC$.
	
	
%==============================================================================================
	
	
\end{document}
