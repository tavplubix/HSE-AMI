\documentclass{article}
\usepackage{cmap}
\usepackage[T2A]{fontenc}
\usepackage[utf8]{inputenc}
\usepackage[english, russian]{babel}
\usepackage[a4paper, left=10mm, right=10mm, top=12mm, bottom=15mm]{geometry}
\usepackage{mathtools,amssymb}
\usepackage{graphicx}
\usepackage{setspace}

\usepackage{listings}

\newenvironment{task}{\begin{center}\fontsize{14}{14}\selectfont\bf}{\rm\fontsize{12}{12}\selectfont\end{center}}

\newcommand{\tch}{\hspace{4px}|\hspace{4px}}
\newcommand{\impl}{\quad \Leftrightarrow \quad}
\newcommand{\rimpl}{\quad \Rightarrow \quad}
\newcommand{\res}[3]{\begin{array}{lcr} #1 & & #2 \\ \hline & #3 \end{array}}
\newcommand{\com}{, \hspace{5px}}
\newcommand{\N}{\mathbb{N}}

\begin{document}
	\begin{center}
		Токмаков Александр, группа БПМИ165 \\
		Домашняя работа 1
	\end{center}

%==============================================================================================
	
	\begin{task} 
		№1
	\end{task}

	В случае, если $m = 0$ возникает неоднозначность: слово $0^p$ может быть записано как $0^01^00^p$ (тогда $f(0^p) = f(0^01^00^p) = 1, \ 0 = 0$) или как $0^q1^00^{p-q}, \ 1 \leq q \leq p$ (тогда $f(0^p) = f(0^q1^00^{p-q}) = 0, \ q \not= 0$). Будем считать, что $f(0^p) = 1$. 
	
	Сначала головка МТ движется вправо до конца слова, затем движется обратно, заменяя '0' на '\#' (пустой символ) до тех пор, пока не встретит '1' или '\#'. Если единиц не встретилось, на ленту записывается ответ '1'. Иначе, головка возвращается к началу слова. Таким образом, на ленте останется слово $0^n1^m$. (Состояния $q_i$).
	
	Затем головка МТ движется вправо до первого символа '0' и заменяет его на 'a'. Если '0' не встретился, головка продолжает двигаться вправо до символа '1' (для какой-то единицы не хватило нуля, $n \not = m$, на ленту записывается ответ '0') либо до символа '\#' (заменено равное количество единиц и нулей, $n = m$, на ленту записывается ответ '1'). Если '0' встретился, движется вправо до первого символа '1' или '\#'. Если встретилась '\#', на ленту записывается ответ '0' (для какого-то нуля не нашлось единицы). Иначе головка перемещается на начало слова, повторяются действия, описанные в этом абзаце. (Состояния $p_i$).
	
	В таблице переходов записаны тройки (новый символ, новое состояние, направление движения); символ '-' в таблице значит, что такая пара (символ, состояние) не может встретиться; начальное состояние $q_{rskip}$; конечные состояния $a$ (accept) и $r$ (reject). 
	\def\arraystretch{1.8}
	\fontsize{14}{14}
	\begin{center}
		$\begin{array}{|c||c|c|c|c|c|}
			\hline 
			\			& \# 				& 0 				& 1 				& a 				& b 				\\ \hline \hline
			q_{rskip} 	& \#, q_{del0}, L	& 0, q_{rskip}, R	& 1, q_{rskip}, R	&	- 				&  -				\\ \hline
			q_{del0} 	& 1, a, L 	 		& \#, q_{del0}, L	& 1, q_{0lskip}, L	&	- 				&  -				\\ \hline
			q_{lskip} 	& \#, p_{repl0}, R 	& 0, q_{lskip}, L	& 1, q_{lskip}, L	&	- 				&  -				\\ \hline \hline
			p_{repl0} 	& -				 	& a, q_{repl1}, R	& 1, s_{find1}, N	& a, p_{repl0}, R 	& b, p_{find1}, N 	\\ \hline
			p_{repl1} 	& \#, r_{rskip}, L 	& - 				& b, p_{lskip}, L	& -				 	& b, p_{repl1}, R 	\\ \hline
			p_{lskip} 	& \#, p_{repl0}, R 	& 0, p_{lskip}, L	& 1, p_{lskip}, L	& a, p_{lskip}, L 	& b, p_{lskip}, L 	\\ \hline
			p_{find1} 	& \#, a_{del}, L 	& -					& 1, r_{rskip}, N	& -				 	& b, p_{find1}, R 	\\ \hline \hline
			r_{rskip} 	& \#, r_{del}, L 	& 0, r_{rskip}, R	& 1, r_{rskip}, R	& a, r_{rskip}, R 	& b, r_{rskip}, R 	\\ \hline
			r_{del} 	& 0, r, L 			& \#, r_{del}, L	& \#, r_{del}, L	& \#, r_{del}, L 	& \#, r_{del}, L 	\\ \hline \hline
			a_{del} 	& 1, a, L 			& \#, a_{del}, L	& \#, a_{del}, L	& \#, a_{del}, L 	& \#, a_{del}, L 	\\ \hline
		\end{array}$
	\end{center}
	\fontsize{12}{12}
	
%==============================================================================================
	\vfill
		
	\begin{task} 
		№2
	\end{task}

	Начальное состояние $q_{rskip}$, конечное состояние $q_{stop}$
	\begin{center}
	\def\arraystretch{1.8}
	\fontsize{14}{14}
	\begin{center}
		$\begin{array}{|c||c|c|c|}
		\hline 
		\			& \# 				& 0 				& 1 				\\ \hline
		q_{rskip} 	& \#, q_{carry}, L	& 0, q_{rskip}, R	& 1, q_{rskip}, R	\\ \hline
		q_{carry} 	& 1, q_{stop}, L 	& 1, q_{lskip}, L	& 1, q_{carry}, L	\\ \hline
		q_{lskip} 	& \#, q_{stop}, N 	& 0, q_{lskip}, L	& 1, q_{lskip}, L	\\ \hline 
		\end{array}$
	\end{center}
	\fontsize{12}{12}
	\end{center}
	\vfill
%==============================================================================================
	\newpage
	
	\begin{task} 
		№3
	\end{task}
	Обозначения:
	\begin{spacing}{1.4}
	\begin{center}
		$Z:\ \N \rightarrow \N, \quad Z(x) = 0 $\\
		$N:\ \N \rightarrow \N, \quad N(x) = x + 1 $\\
		$U_i^n:\ \N^n \rightarrow \N, \quad U(x_1, ..., x_n) = x_i $\\
		$S\langle f, g\rangle :\ \N^m \rightarrow \N, \quad S\langle f, g\rangle(x_1, ..., x_m) = f(g(x_1, ..., x_m), ..., g(x_1, ..., x_m)) $, где $f:\ \N^n \rightarrow \N,\ g:\ \N^m \rightarrow \N$\\
		$R\langle f, g\rangle :\ \N^{n+1} \rightarrow \N \quad R\langle f, g\rangle(x_1, ..., x_n, y) = \begin{cases}
		f(x_1, ..., x_n) & if \ y = 0 \\
		g(x_1, ..., x_n, y - 1, R\langle f, g\rangle(x_1, ..., x_n, y-1)) & else
		\end{cases}$ \\ где $f:\ \N^n \rightarrow \N,\ g:\ \N^{n+2} \rightarrow \N$\\
	\end{center}

	Определим вспомогательные функции (все они примитивно рекурсивные по построению):\\
	Декремент $i$-того аргумента: 
	\begin{center}
		$Dec(x) = S \langle R\langle Z, U_2^3 \rangle, U_1^1, U_1^1 \rangle(x) = 
		R\langle Z, U_2^3 \rangle (x, x) = 	
		\begin{cases}
		0 & if \ x = 0 \\
		x - 1 & else
		\end{cases}$\\
		$Dec_i^n(x_1, ..., x_n) = S\langle Dec, U_i^n \rangle (x_1, ..., x_n) = Dec(x_i)$
	\end{center}
	Вычитание:
	\begin{center}
		$Sub(x, y) = R\langle U_1^1, Dec_3^3 \rangle (x, y) = \begin{cases}
		0 & if \ x \leq y \\
		x - y & else
		\end{cases}$\\
	\end{center}
	Единица от $n$ аргументов:
	\begin{center}
		$One^n(x_1, ..., x_n) = S\langle S\langle N, Z \rangle, U_1^n \rangle(x_1, ..., x_n) = N(Z(U_1^n(x_1, ..., x_n))) = 1$\\
	\end{center}
	Сравнение с нулём:
	\begin{center}
		$EqZ(x) = R\langle Z, One^2 \rangle (x) = \begin{cases}
		0 & if \ x = 0 \\
		1 & else
		\end{cases}$\\
	\end{center}
	Сравнения $x \leq y$ и $x \geq y$: 
	\begin{center}
		$Leq(x, y) = S\langle EqZ, Sub \rangle (x, y) = EqZ(x - y) = \begin{cases}
		0 & if \ x \leq y \\
		1 & else
		\end{cases}$\\
		$Geq(x, y) = S\langle Leq, U_2^2, U_1^2 \rangle (x, y) = EqZ(y - x) = \begin{cases}
		0 & if \ x \geq y \\
		1 & else
		\end{cases}$\\
	\end{center}
	Инкремент $i$-того аргумента: 
	\begin{center}
		$Inc_i^n(x_1, ..., x_n) = S\langle N, U_i^n \rangle (x_1, ..., x_n) = N(x_i) = x_i + 1$ \\
	\end{center}
	Сумма:
	\begin{center}
		$Sum(x, y) = R\langle U_1^1, Inc_3^3 \rangle (x, y) = x + y$\\
	\end{center}
	Сравнение $x = y$:
	\begin{center}
		$Eq(x, y) = S\langle Sum, Leq, Geq \rangle (x, y) = Leq(x, y) + Geq(x, y) = \begin{cases}
		0 & if \ x = y \\
		1 & else
		\end{cases}$\\
	\end{center}
	Поиск значения $f(i) = c$ для $0 \leq i \leq y$:
	\begin{center}
		$Cmp_f(c, i) = S\langle Eq, U_1^2, S \langle f, U_2^2 \rangle\rangle (c, i) =  
		Eq(c, f(i)) = 
		\begin{cases}
		0 & if \ f(i) = c \\
		1 & else
		\end{cases}$\\
		$SumCmp_f(c, i, s) = S\langle Sum, S \langle Cmp_f, U_1^3, U_2^3 \rangle, U_3^3 \rangle(c, i) =  
		Sum(Eq(c, f(i)), s) = 
		\begin{cases}
		s & if \ f(i) = c \\
		s + 1 & else
		\end{cases}$\\
		$Find_f(c, y) = R\langle Z, SumCmp_f\rangle (x, y) = \lvert\{ i \in \mathbb{N} : \ i \leq y \ \wedge  \ f(i) = c \}\rvert$
	\end{center}
	Подсчёт пар $(i, j), \ 0 \leq i, j \leq y$ , для которых $g(i) = h(i)$:
	\begin{center}
		$FindVal_{g,h}(j, y) = S\langle Find_g, S\langle h, U_1^2\rangle, U_2^2\rangle (j, y) =  
		Find_g(h(i), y) = \lvert\{ i \in \mathbb{N} : \ i \leq y \ \wedge  \ g(i) = h(j) \}\rvert$\\
		$SumFindVal_{g,h}(y, j, s) = S\langle Sum, S \langle FindVal_{g,h}, U_2^3, U_1^3 \rangle, U_3^3 \rangle(c, i) =  $\\$=
		Sum(FindVal_{g,h}(j, y), s) = s + \lvert\{ i \in \mathbb{N} : \ i \leq y \ \wedge  \ g(i) = h(j) \}\rvert$\\
		$FindPairs(x, y) = R\langle Z, SumFindVal_{g, h}\rangle (x, y) = \lvert\{ (i, j) \in \mathbb{N}^2 : \ i \leq x \ \wedge \ j \leq y \ \wedge  \ g(i) = h(j) \}\rvert$
		$F(y) = S \langle FindPairs, U_1^1, U_1^1 \rangle = FindPairs(y, y) =$\\$= \lvert\{ (i, j) \in \mathbb{N}^2 : \ i,j \leq y \ \wedge  \ g(i) = h(j) \}\rvert = \begin{cases}
		0 & if \ g(i) \not = h(j)\ for\ 0 \leq i, j \leq y \\
		c > 0 & else
		\end{cases}$
	\end{center}
	Осталось сделать так, чтобы $F(y)$ возвращала только 0 или 1: 
	\begin{center}
	$F_1(y) = S\langle EqZ, F \rangle (y) = EqZ(F(y)) =  \begin{cases}
	0 & if \ g(i) \not = h(j)\ for\ 0 \leq i, j \leq y \\
	1 & else
	\end{cases}$ 
	\end{center}
	$F_1(y)$ -- искомая функция, если $h$ и $g$ примитивно рекурсивны, то $F_1$ примитивно рекурсивна по построению. 
 	\end{spacing}
	
%==============================================================================================
		
	\begin{task} 
		№4
	\end{task}

	Здесь используются обозначения и некоторые функции из №3.\\
	Определим умножение:
	\begin{center}
		$Mul(x, y) = R\langle Z, S\langle Sum, U_1^3, U_3^3 \rangle \rangle = x\cdot y = \begin{cases}
		Z(x) & if \ y = 0 \\
		Sum(x, Mul(x, y-1)) & else
		\end{cases}$
	\end{center}
	Инвертируем значения $Leq$:
	\begin{center}
		$Leq_1(x, y) = S\langle Sub, One^2, Leq \rangle = 1 - Leq(x, y) = \begin{cases}
		1 & if \ x \leq y \\
		0 & else
		\end{cases}$
	\end{center}
	Заметим, что $\lfloor \frac{x}{y} \rfloor = z \ \iff \ (z + 1)\cdot y > x \geq z\cdot y$ и $0 \leq z \leq x$. Таким образом, чтобы найти $z$, можно перебирать числа от 0 до $x$ до тех пор, пока не начнёт выполняться условие $(z + 1)\cdot y > x$:
	\begin{lstlisting}
	def div(x, y):
	    for z in from 0 to x:
	        if (z + 1) * y > x:
	            return z
	    return 0	# let div(x, y) = 0 if y = 0
	\end{lstlisting}
	Запишем этот же алгоритм в другом виде:
	\begin{lstlisting}
	def div(x, y):
	    z = 0
	    for i in from 0 to x:
	        if (z + 1) * y > x:
	            z = z
	        else
	            z = z + 1
	    retrun z
	\end{lstlisting}
	И ещё раз:
	\begin{lstlisting}
	def div(x, y):
	    z = 0
	    for i in from 0 to x:
	        z = z + Leq_1((z+1)*y, x)
	    retrun z
	\end{lstlisting}
	Такой алгоритм уже легко записать с помощью примитивно рекурсивных функций:
	
	\begin{spacing}{1.4}
	\begin{center}
		$MulP1^4(x, y, c, z) = S\langle Mul, S\langle N, U_4^4 \rangle, U_2^4 \rangle(x, y, c, z) = (z + 1)\cdot y$\\
		$Cond^4(x, y, c, z) = Leq_1(MulP1^4(x, y, c, z), x) = S\langle Leq_1, MulP1^4, U_1^4 \rangle(x, y, c, z)$\\
		$g(x, y, c, z) = Sum(z, Leq_1(y\cdot(z+1), x)) = S\langle Sum, U_4^4, Cond^4 \rangle(x, y, c, z)$\\
		$FindZ(x, y, c) = R\langle S\langle Z, U_1^2 \rangle g \rangle(x, y, c)$ \\
		$Div(x, y) = FindZ(x, y, x) =  S\langle FindZ, U_1^3, U_2^3, U_1^3 \rangle(x, y)$\\
		$f(x, y) = S\langle Div, U_1^2, S\langle N, U_2^2 \rangle\rangle = Div(x, y + 1) = \lfloor \frac{x}{y+1} \rfloor$
	\end{center}
	\end{spacing}
	Функция $f(x, y) = \lfloor \frac{x}{y+1} \rfloor$ примитивно рекурсивна.
	
	
%==============================================================================================
	
	
			
	\begin{task} 
		№5
	\end{task}

	Для удобства вместо номеров строк будем использовать метки, как в ассемблере (строка, заканчивающаяся двоеточием, указывает на следующую за ней команду). По метке можно легко получить номер команды.\\
	Обозначения:\\
	
	\hspace{30px}	$T(n, m)$ -- скопировать значение из регистра с номером $n$ в регистр с номером $m$ \\
		
	\hspace{30px}	$Z(n)$ -- обнулить регистр с номером $n$ \\
		
	\hspace{30px}	$S(n)$ -- увеличить на 1 значение в регистре с номером $n$ \\
		
	\hspace{30px}	$J(n, m, $LABEL$)$ -- если значение в регистре с номером $n$ равно значению в регистре с номером $m$, перейти
	\hspace{30px} к строке с меткой LABEL, иначе - к следующей строке \\
		
	\hspace{30px}	Аргумент вычисляемой функции и вычисленный результат находятся в нулевом регистре.\\
	Программа вычисляет $f(x) = \begin{cases}
	x + 1 & if \ x \mod 2 = 0 \\
	x - 1 & else
	\end{cases}$
	\begin{lstlisting}
		Z(1)
	LOOP1:
		J(0, 1, EVEN)
		T(1, 2)
		S(1)
		J(0, 1, ODD)
		S(1)
		J(0, 0, LOOP1)
	EVEN:
		S(0)
		J(0, 0, END)
	ODD:
		T(2, 0)
	END:
		T(0, 0)
	\end{lstlisting}


%==============================================================================================	


	\begin{task} 
		№6
	\end{task}
	Обозначения:\\
	
	\hspace{30px}	$M(n, m, p)$ \ -- \ $r_n = r_m \cdot r_p$
	
	Программа вычисляет $f(x, y) = \lfloor \frac{x}{y+1} \rfloor$. Аргумент $x$ передаётся в нулевом регистре, $y$ -- в первом регистре, результат -- в нулевом регистре.
	
	\begin{lstlisting}
		S(1)
		Z(2)			// r2 is z = f(x, y)
	LOOP:
		T(2, 3)
		S(3)
		M(4, 1, 2)		// r4 = z * (y + 1)
		M(5, 1, 3)		// r5 = (z + 1) * (y + 1)
		// Let's try to find x (r0) in range between r4 and r5
		T(4, 6)
	FINDX:
		J(0, 6, FOUND)
		S(6)
		J(5, 6, NOT_FOUND)
		J(0, 0, FINDX)
	NOT_FOUND:			// Let's try whith next z
		S(2)
		J(0, 0, LOOP)
	FOUND:		// z * (y + 1) <= x < (z + 1)*(y + 1), so f(x, y) = z
		T(2, 0)
	\end{lstlisting}
	
%==============================================================================================
	
	
\end{document}
