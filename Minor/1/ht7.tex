\documentclass{article}
\usepackage{cmap}
\usepackage[T2A]{fontenc}
\usepackage[utf8]{inputenc}
\usepackage[english, russian]{babel}
\usepackage[a4paper, left=10mm, right=10mm, top=12mm, bottom=15mm]{geometry}
\usepackage{mathtools,amssymb}
\usepackage{ulem}
\usepackage{graphicx}
\newcommand{\bigslant}[2]{{\raisebox{.2em}{$#1$}\left/\raisebox{-.2em}{$#2$}\right.}}


\linespread{1.4}

\newenvironment{task}{\begin{center}\fontsize{14}{14}\selectfont\bf}{\rm\fontsize{12}{12}\selectfont\end{center}}

\newcommand{\cvec}[1]{\left(\begin{array}{c} #1 \end{array}\right)}

\newcommand{\impl}{\quad\Leftrightarrow\quad}
\newcommand{\rimpl}{\quad\Rightarrow\quad}
\newcommand{\Z}{\mathbb{Z}}
\newcommand{\N}{\mathbb{N}}

\begin{document}
	\begin{center}
		Токмаков Александр, ФКН, группа БПМИ165 \\
		Домашнее задание 7
	\end{center}
	
	\begin{task} 
		№1
	\end{task}
	Отрезок, проходящий через середины двух сторон треугольника является его средней линией и равен половине третьей стороны. В данной задаче маленький треугольник состоит из средних линий большого треугольника, значит каждая его сторона в два раза меньше, значит его периметр тоже в два раза меньше и равен $\frac{28}{2} = 14$.

%===================================================================================================================


	\begin{task} 
		№2
	\end{task}
	\begin{center}
		\includegraphics[height=8cm]{2}\\
		$ABCD$ -- произвольный четырёхугольник \\
		$E$ -- точка пересечения диагоналей \\
		$F$ -- произвольная точка
	\end{center}
	Докажем, что сумма расстояний $AE + EC + DE + EB$ от вершин $ABCD$ до $E$ минимальна. Пусть для некоторой точки $F$ сумма расстояний $AF + FC + DF + FB$ меньше. Запишем неравенства треугольника для $\triangle AFC$ и $\triangle DFB$: 
	\begin{center}
		$AF + FC \geq AC$ \\
		$DF + FB \geq DB$
	\end{center}
	Сложим эти неравенства:
	\begin{center}
		$AF + FC + DF + FB \geq AC + DB = AE + EC + DE + EB$ 
	\end{center}
	Таким образом, сумма расстояний до произвольной точки $F$ не меньше суммы расстояний до $E$, значит сумма расстояний от вершин до точки $E$ минимальна. Причём неравенства насыщаются только при $F \in AC$ и $F \in BD$ т.е. $F=E$, значит такая точка единственна. 
	
	
%===================================================================================================================
	\newpage
	\begin{center}
		Токмаков Александр, ФКН, группа БПМИ165 \\
		Домашнее задание 7
	\end{center}
	
	\begin{task} 
		№3
	\end{task}
	\begin{center}
		\includegraphics[height=12cm]{3}\\
		$TOP$ -- произвольный острый или прямой угол \\
		$M$ -- произвольная точка внутри угла\\
		$MC \bot OT, \quad MD \bot OP$ -- перпендикуляры из $M$ к сторонам угла \\
		$KC = CM, \quad ND=DM$ -- точки $K$ и $M$ симметричны относительно $OT$, $N$ и $M$ -- относительно $OP$ \\
		$A = KN \cap OT, \quad B = KN \cap OP$ - искомые точки \\
		$P_{\triangle MAB} = MA + AB + BM = KA + AB + BN$ \\
		$F, H$ -- некоторые точки на сторонах угла 
	\end{center}
	Докажем, что $P_{\triangle MAB}$ не может быть меньше:
	\begin{center}
		$P_{\triangle MAB} = KA + AB + BN = KB + BN \leq KH + HN \leq KF + FH + HN \quad \forall F \forall H$ 
	\end{center}
	Если угол тупой, то $A = B = O$.
	
	 
%===================================================================================================================
	

	\begin{task} 
		№4
	\end{task}
	Диагональ разбивает четырёхугольник на два треугольника. Длина диагонали не может равняться 7.5 т.к. неравенство треугольника выполняется только для $7.5 \leq 5 + 2.8$, но тогда для другой пары сторон получится $7.5 \leq 1 + 2$. Также длина диагонали не может равняться 5 т.к. даже если парой стороны с длиной 7.5 является наименьшая сторона длины 1, то для другой пары сторон получится $5 \leq 2 + 2.8$. 
    Стороны длины 1 и 2 тоже не могут быть диагоналями т.к. $7.5 - 1$ и $7.5 - 2$ больше всех остальных сторон. Значит, диагональ равна 2.8; такое возможно:
    \begin{center}
    	$1 + 2 > 2.8, \quad 1 + 2.8 > 2, \quad 2.8 + 2 > 1$ \\
    	$5 + 7.5 > 2.8, \quad 7.5 + 2.8 > 5, \quad 2.8 + 5 > 7$ 
    \end{center}
     
	
	\newpage
	\begin{center}
		Токмаков Александр, ФКН, группа БПМИ165 \\
		Домашнее задание 7
	\end{center}
%===================================================================================================================
	
	\begin{task} 
		№5
	\end{task}
	Точка, наименее удалённая от вершин четырёхугольника -- пересечение его диагоналей. Значит, длина кратчайшей системы дорог равна сумме длин диагоналей и равна $2\cdot4\cdot\sqrt{2} > 11 \quad (128 > 121)$. Такую систему дорог построить нельзя. 
	
	
%===================================================================================================================
	
	\begin{task} 
		№6
	\end{task}
	Не верно. Рассмотрим $\triangle ABC$ со сторонами $ AB = 2, \quad BC = 3, \quad AC = \frac{9}{2}$ (такой треугольник существует) и подобный ему $\triangle A_1B_1C_1$ с коэффициентом $\frac{3}{2}, \quad A_1B_1 = 3, \quad B_1C_1 = \frac{9}{2}, \quad A_1C_1 = \frac{27}{4}$. Три угла одного треугольника равны трём углам другого, и две стороны одного равны двум сторонам другого, но третьи стороны разные, значит треугольники не равны. 
	
	
	%===================================================================================================================
	
	










\end{document}
