\documentclass{article}
\usepackage{cmap}
\usepackage[T2A]{fontenc}
\usepackage[utf8]{inputenc}
\usepackage[english, russian]{babel}
\usepackage[a4paper, left=10mm, right=10mm, top=12mm, bottom=15mm]{geometry}
\usepackage{mathtools,amssymb}
\newcommand{\bigslant}[2]{{\raisebox{.2em}{$#1$}\left/\raisebox{-.2em}{$#2$}\right.}}

\linespread{1.4}

\newenvironment{task}{\begin{center}\fontsize{14}{14}\selectfont\bf}{\rm\fontsize{12}{12}\selectfont\end{center}}


\begin{document}
	\begin{center}
		Токмаков Александр, ФКН, группа БПМИ165 \\
		Домашнее задание 3
	\end{center}
	
	\begin{task} 
		№1
	\end{task}
	\begin{center}
		$0 \cdot 1 + 1\cdot 2 + 2\cdot 3 + 3\cdot4 + \dots + (n-1)n = \frac{(n-1)n(n+1)}{3} \quad \forall n\in \mathbb{N}$\\
	\end{center}
	Мне лень придумывать что-то другое, поэтому докажем по индукции. \\
	База: при $n=1$ получим верное равенство $0\cdot1 =\frac{(1-1)\cdot 1\cdot (1+1)}{3}$\\
	Пусть равенство верно для $n \in \mathbb{N}$, тогда для $n+1$:
	\begin{center}
		$0 \cdot 1 + 1\cdot 2 + 2\cdot 3 + 3\cdot4 + \dots + (n-1)n + n(n+1) = \frac{(n)(n+1)(n+2)}{3}$\\
		$0 \cdot 1 + 1\cdot 2 + 2\cdot 3 + 3\cdot4 + \dots + (n-1)n = \frac{(n)(n+1)(n+2)}{3} - \frac{3n(n+1)}{3}$\\
		$0 \cdot 1 + 1\cdot 2 + 2\cdot 3 + 3\cdot4 + \dots + (n-1)n = (n+1)\left( \frac{(n)(n+2)}{3} - \frac{3n}{3} \right)$\\
		$0 \cdot 1 + 1\cdot 2 + 2\cdot 3 + 3\cdot4 + \dots + (n-1)n = (n+1)\frac{n^2 - n}{3}$\\
		
		$0 \cdot 1 + 1\cdot 2 + 2\cdot 3 + 3\cdot4 + \dots + (n-1)n = (n+1)\frac{n(n-1)}{3}$\\
	\end{center} 
	Выполнив эти преобразования в обратном порядке (снизу вверх), получим, что из равенства для $n$ следует равенство для $n+1$. Значит, по аксиоме индукции, равенство верно для любого натурально
	го $n$.
	
	\begin{task} 
		№2
	\end{task}
	\begin{center}
		$2^n > n \quad \forall n \in \mathbb{N}$
	\end{center}
	Индукция уже была, использовать её снова не интересно. Докажем более сильное утверждение с помощью матана:
	\begin{center}
		$2^x > x  \quad \forall x \in \mathbb{R} \supset \mathbb{N}$
	\end{center}
	Рассмотрим функцию $f(x) = 2^x - x$. Найдём её локальный экстремум, и покажем, что это минимум, он единственный и функция в этой точке положительна:
	\begin{center}
		$\frac{df(x)}{dx} = \ln(2)\cdot 2^x - 1 = 0$ \\
		$2^x = \frac{1}{\ln(2)}$ \\
		$x_{min} = \log_2\left(\frac{1}{\ln(2)}\right) = -\log_2(\ln(2))$ \\
		$f(x_{min}) = f(-\log_2(\ln(2))) = 2^{-\log_2(\ln(2))} + \log_2(\ln(2)) = \frac{1}{\ln(2)} + \log_2(\ln(2))$ \\
		\begin{tabular}{r|}
			$2 < e \Rightarrow \ln(2) < 1 \Rightarrow 1 < \frac{1}{\ln(2)}$ \\
			$e < 2^2 \Rightarrow \frac{1}{2} < \ln(2) \Rightarrow -1 < \log_2(\ln(2))$ \\
		\end{tabular} $\quad \Rightarrow \quad f(x_{min}) = \frac{1}{\ln(2)} + \log_2(\ln(2)) > 0$
		
		
	\end{center}
	Экстремум единственный т.к. точка $x_{min}$ задаётся явной формулой. \\
	Вторая производная $\frac{d^2f(x)}{dx^2} = \ln^2(2)\cdot 2^x > 0 \Rightarrow$ первая производная монотонно возрастает. Также первая производная непрерывна как сумма непрерывных функций. Следовательно, $\frac{df(x)}{dx} \leq 0 $ при $x \leq x_{min}$ и $0 \leq \frac{df(x)}{dx}$ при $x_{min} \leq x \Rightarrow$ экстремум - минимум. \\
	Функция $f(x)$ непрерывна, имеет единственный локальный минимум $ 0 < f(x_{min})$, убывает при $x \leq x_{min}$ и возрастает при x $\leq x_{min}$, следовательно, локальный минимум $ 0 < f(x_{min})$ является глобальным, следовательно \\ \begin{center} $0 < f(x) = 2^x-x \quad\Rightarrow\quad 2^x > x$ \end{center}
	


	\begin{task} 
		№3
	\end{task}
	Очевидно, что среди любых четырёх последовательных целых чисел встретиться ровно два чётных числа (ровно одно из которых при этом делится на 4) и как минимум одно число, делящееся на 3. Тогда произведение этих чисел делится на $2\cdot4\cdot3=24$. \\


	\begin{task} 
		№4
	\end{task}
	\begin{center}
		$1 + \frac{1}{2^1} + \frac{1}{2^2} + \dots + \frac{1}{2^n} < 2 \quad \forall n \in \mathbb{N}$
	\end{center}
	Заметим, что слева стоит сумма первых $n+1$ членов геометрической прогрессии с первым членом $b = 1$ и знаменателем $q=\frac{1}{2}$:
	\begin{center}
		$1 + \frac{1}{2^1} + \frac{1}{2^2} + \dots + \frac{1}{2^n} = \frac{b(1-q)}{1 - q^{n+1}} = \frac{\frac{1}{2}}{1 - \frac{1}{2^{n+1}}}$\\
		$\frac{\frac{1}{2}}{1 - \frac{1}{2\cdot2^n}} < 2		
		\quad\Leftrightarrow\quad
		\frac{1}{\frac{2\cdot2^n - 1}{2\cdot2^n}} < 4
		\quad\Leftrightarrow\quad
		\frac{2^n}{2\cdot2^n - 1} < 2$ \\
		$2^n > n \quad\forall n \in \mathbb{N} \quad\Rightarrow\quad 2^n > n \geq 1 \quad\Rightarrow\quad 2\cdot 2^n > 1 \quad\Rightarrow\quad 2\cdot2^n-1 > 0$\\
		можно домножить неравенство на $2\cdot2^n-1$\\
		$2^n < 4\cdot2^n - 2
		\quad\Leftrightarrow\quad
		3\cdot 2^n > 2
		\quad\Leftrightarrow\quad
		2^n > 1 > \frac{2}{3}$ 
	\end{center}
	Равносильными преобразованиями получено верное неравенство, значит исходное неравенство верно.

	\begin{task} 
		№5
	\end{task}
	Пусть $f(n)$ - количество частей, на которые разделят плоскость $n$ окружностей.\\
	Очевидно, что $f(1) = 2$ (одна окружность делит плоскость на две части). Пусть проведено $n$ окружностей. Проведём ещё одну. Она пересечёт каждую из уже имевшихся окружностей ровно в двух точках, причём ни одна из новых точек пересечения не совпадёт ни с одной из старых т.к. иначе есть 3 окружности, проходящие через одну точку. Таким образом, новая окружность создаст $2n$ новых точек пересечения, которые разделят её на $2n$ дуг. Каждая дуга разобьёт некоторую область на две, т.е. появятся $2n$ новых областей. \\
	Получаем рекуррентное соотношение $f(n+1) = f(n) + 2n$ \\
	Заметим, что $f(n) = 2 + 2 + 4 + 6 + 8 + 10 + \dots + 2(n-1) 
	= 2 + 2(1 + 2 + 3 + \dots + n-1) 
	= 2 + 2\cdot \frac{(n-1)n}{2} = 2 + (n-1)n$
	Действительно, $f(n+1) = 2 + n(n+1) = 2 + n^2 - n + 2n = 2 + (n-1)n + 2n = f(n) + 2n$  \\
	Таким образом, $n$ окружностей разделят плоскость на $2 + (n-1)n$ частей.

	\begin{task} 
		№6
	\end{task}
	Построим поле из четырёх элементов :
	\begin{center}
		$p(x) = x^2 + x + 1$ - неприводимый над $\mathbb{Z}_2$ многочлен степени 2\\
		(т.к. $p(0) \not = 0$ и $p(1) \not = 0$ его нельзя разложить на множители)\\
		$\mathbb{F}_4 \simeq \bigslant{\mathbb{Z}_2[x]}{(x^2 + x + 1)}$  \\
		$\phi: \mathbb{Z}_2[x] \rightarrow \mathbb{Z}_2[x]_{<2}$, 
		$\quad \phi(a) = a \hspace{5px}mod\hspace{5px}p$ \\
		$\phi(ab) = ab \hspace{5px}mod\hspace{5px}p 
		= (a \hspace{5px}mod\hspace{5px}p)(b \hspace{5px}mod\hspace{5px}p) \hspace{5px}mod\hspace{5px}p 
		= \phi(a)\phi(b) \hspace{5px}mod\hspace{5px}p
		= \phi(a)\phi(b) $\\
		$\phi(a+b) = (a+b) \hspace{5px}mod\hspace{5px}p 
		= (a \hspace{5px}mod\hspace{5px}p) + (b\hspace{5px}mod\hspace{5px}p) =
		\phi(a)+\phi(b)$ \\
		т.е. $\phi$ - гомоморфизм колец \\
		$Ker\phi = \left\lbrace a \in \mathbb{Z}_2[x] \hspace{5px} | \hspace{5px} a \hspace{5px}\vdots\hspace{5px} p\right\rbrace = \left\lbrace a \in \mathbb{Z}_2[x] \hspace{5px} | \hspace{5px} \exists b \in \mathbb{Z}_2[x] \hspace{5px} | \hspace{5px}  b\cdot p = a\right\rbrace = (x^2 + x + 1)$ \\ $\quad Im\phi = \mathbb{Z}_2[x]_{<2} $
	\end{center}
	По теореме о гомоморфизме колец:
	\begin{center}
		$Im\phi = \mathbb{Z}_2[x]_{<2} \simeq \bigslant{\mathbb{Z}_2[x]}{(x^2 + x + 1)}$  \\
		$\mathbb{F}_4 \simeq \mathbb{Z}_2[x]_{<2}$ \\
		$\mathbb{F}_4 = \left\lbrace 0, 1, x, x+1 \right\rbrace$
	\end{center}
	Операции в $\mathbb{F}_4$ - сложение и умножение многочленов по модулю $x^2+x+1$ \\
	Докажем, что если $p$ - простое число, то $\mathbb{Z}_p$ - поле:
	\begin{center}
		пусть $a \in \mathbb{Z}_p \backslash \lbrace0\rbrace$, найдём $a^{-1}$\\
		$p$ - простое и $a < p \quad\Rightarrow\quad$ НОД$(a, p) = 1 
		\quad\Rightarrow\quad 
		\exists m, n \hspace{5px} | \hspace{5px} ma+np=1$ \\ $
		\quad\Rightarrow\quad 
		 ma+np\equiv1 \hspace{5px}mod\hspace{5px}p 
		 \quad\Rightarrow\quad 
		 ma\equiv1 \hspace{5px}mod\hspace{5px}p 
		$ 
	\end{center}
	Таким образом, $\mathbb{F}_5 = \mathbb{Z}_5$ с операциями сложения и умножения по модулю 5. \\
	
	Поля из $6 = 2\cdot 3$ элементов не бывает, потому что в конечном поле всегда $p^n$ элементов, где $p$ - простое, $n \in \mathbb{N}$
	 
	




\end{document}