\documentclass{article}
\usepackage{cmap}
\usepackage[T2A]{fontenc}
\usepackage[utf8]{inputenc}
\usepackage[english, russian]{babel}
\usepackage[a4paper, left=10mm, right=10mm, top=12mm, bottom=15mm]{geometry}
\usepackage{mathtools,amssymb}
\usepackage{graphicx}
\usepackage{setspace}

\usepackage{listings}

\newenvironment{task}{\begin{center}\fontsize{14}{14}\selectfont\bf}{\rm\fontsize{12}{12}\selectfont\end{center}}

\newcommand{\tch}{\hspace{4px}|\hspace{4px}}
\newcommand{\impl}{\quad \Leftrightarrow \quad}
\newcommand{\rimpl}{\quad \Rightarrow \quad}
\newcommand{\res}[3]{\begin{array}{lcr} #1 & & #2 \\ \hline & #3 \end{array}}
\newcommand{\com}{, \hspace{5px}}
\newcommand{\N}{\mathbb{N}}

\begin{document}
	\begin{center}
		Токмаков Александр, группа БПМИ165 \\
		Домашняя работа 1
	\end{center}

%==============================================================================================
	
	\begin{task} 
		№1
	\end{task}

	Построим таблицу истинности для формулы $(p \vee q) \rightarrow (p \vee \overline{r})$:
	\begin{center}
			$\begin{array}{|c|c|c|c|}
		\hline 
		p			& q 				& r 				&  			\\ \hline
		0			& 0 				& 0 				& 1 		\\ \hline
		0			& 0 				& 1 				& 1 		\\ \hline
		0			& 1 				& 0 				& 1 		\\ \hline
		0			& 1 				& 1 				& 0 		\\ \hline
		1			& 0 				& 0 				& 1 		\\ \hline
		1			& 0 				& 1 				& 1 		\\ \hline
		1			& 1 				& 0 				& 1 		\\ \hline
		1			& 1 				& 1 				& 1 		\\ \hline
		
		\end{array}$
	\end{center}
	
	По ней видно, что ДНФ будет содержать все дизъюнкты, кроме $\overline{p} \vee q \vee r$: 

	\begin{center}
		$ (\overline{p} \wedge \overline{q} \wedge \overline{r}) \vee (\overline{p} \wedge \overline{q} \wedge r) \vee (\overline{p} \wedge q \wedge \overline{r}) \vee 
		  (p \wedge \overline{q} \wedge \overline{r}) \vee (p \wedge \overline{q} \wedge r) \vee (p \wedge q \wedge \overline{r}) \vee (p \wedge q \wedge r)$
	\end{center}
	\fontsize{12}{12}
	
%==============================================================================================

		
	\begin{task} 
		№2
	\end{task}

	Если выражение, стоящее слева от импликации ложно, то формула истинна. Если оно истинно, то для любого $i$ истинно $\bigwedge_{j=1}^{n} p_{ij}$, т.е. в каждой строке матрицы $p$ есть хотя бы одна единица. Т.к. в матрице $n$ столбцов $n+1$ строк (строк больше, чем столбцов), по принципу Дирихле найдутся хотя бы 2 строки $i_1, i_2$, в которых единицы стоят в одинаковых столбцах $j$. Таким образом, найдутся такие $j, i_1, i_2$, что $(p_{i_{1}j} \wedge p_{i_{2}j})$ истинно, значит всё выражение, стоящее справа от импликации, истинно. Значит, формула истинна для любых значений переменных т.е. является тавтологией.
	

	
\end{document}
